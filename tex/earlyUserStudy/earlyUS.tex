\documentclass[11pt]{amsart}
\usepackage{geometry}                % See geometry.pdf to learn the layout options. There are lots.
\geometry{letterpaper}                   % ... or a4paper or a5paper or ... 
%\geometry{landscape}                % Activate for for rotated page geometry
%\usepackage[parfill]{parskip}    % Activate to begin paragraphs with an empty line rather than an indent
\usepackage{graphicx}
\usepackage{amssymb}
\usepackage{epstopdf}
\DeclareGraphicsRule{.tif}{png}{.png}{`convert #1 `dirname #1`/`basename #1 .tif`.png}

\title{Displex early user study}
\author{lnsi}
%\date{}                                           % Activate to display a given date or no date

\begin{document}
\maketitle

\section{Introduction}
``You are here to participate in a short usability study whose purpose is to assist in the design of a software application called Displex.
The experiment will last about 30 minutes and is being recorded, both as audio and video.
I will give you instructions by reading from this script, so as to give you the same information as to the other participants.
After this introduction you will be able to ask questions before we start the experiment.

First, let me introduce the application.
Displex is a software application that connects a smartphone (in this case an iPhone) with a Microsoft Surface tabletop computer.
It allows you to interact with your smartphone on a larger screen, by transfering the display of your smartphone to the screen of the interactive surface.
Do you understand the basic concept of the application?

During this experiment, I will ask you to perform a task using the application.
This will lead you to perform a number of actions that use the basic features of Displex.
For each action, there will be 3 steps:

\begin{itemize}
\item{First, I will explain the action, and show you its effect on this screen}
\item{Second, you will describe to me how you would suggest performing this action with the user interface of Displex.}
\item{Third, I will describe 3 different ways of performing the action, and ask you to order the suggestions by order of your preference.}
\end{itemize}

This experiment is based on prototypes, meaning that we will use the available paper representations in order to describe the user interface of Displex.
There are paper, pen and scissors available for building your own prototypes if necessary.
We will also use the iPhone, the MS Surface, and of course words.

Do you have any questions concerning the general course of the experiment?
\\\\
Let us begin.
Your general task is to write an email to a friend using your iPhone and the Displex application on the Microsoft Surface.
We will talk about 7 basic actions.''

\pagebreak

\section{Experiment}

\subsection{Pairing}
\hfill\\
This first action is an example, meaning that I will go through all the steps myself.
The action is called pairing.
In order to use Displex, I have to pair my iPhone with the Surface and launch the Displex application. Here is a visual representation of the effect of this application.
(SHOW PPF)
\\\\
\emph{Open suggestion??}\\
I launch Displex on my iPhone, search the local network for available Surface computers, and connect to the Surface.
\\\\
\emph{Guided suggestions}
\begin{description}
\item[A]{The application launches automatically when the smartphone is placed on the surface, and a dialog window appears on the smartphone, offering the user to establish the connection.}
\item[B]{The application launches automatically when the smartphone is placed on the surface, and 2 dialog windows appear, first on the surface, then on the smartphone, offering the user to establish the connection.}
\item[C]{The application launches automatically when the smartphone is close enough to the surface, and a dialog window appears on the surface, offering the user to establish the connection.}
\end{description}

My order of preference is B, A, C.

\subsection{Dragging}
\hfill\\
Your iPhone screen is now active on the surface, and you need to move it closer to yourself. Therefore, you drag the window across the surface.
(SHOW PPF)
\\\\
\emph{Open suggestion}
\\\\
\emph{Guided suggestions}
\begin{description}
\item[A]{The user performs a one finger dragging gesture on a specific tab.}
\item[B]{The user performs a one finger dragging gesture on the active border of the window.}
\item[C]{The user taps a tab to render the window inactive, then performs a one finger dragging gesture anywhere on the window.}
\end{description}

\subsection{Rotating}
\hfill\\
the application window is not oriented correctly, so you need to rotate it to the correct orientation. 
(SHOW PPF)
\\\\
\emph{Open suggestion}
\\\\
\emph{Guided suggestions}
\begin{description}
\item[A]{The user performs a two finger touch rotating gesture with one finger placed on a specific tab, and the other finger anywhere on the window.}
\item[B]{The user performs a two finger touch rotating gesture on a large tab.}
\item[C]{The user taps a tab to render the window inactive, then performs a two finger touch rotating gesture anywhere on the window.}
\end{description}

\subsection{Resizing}
\hfill\\
Now you open the Safari App by taping on the correct icon, but the window is too small for you to type an email, so you resize it to make it bigger.
(SHOW PPF)
\\\\
\emph{Open suggestion}
\\\\
\emph{Guided suggestions}
\begin{description}
\item[A]{The user performs a one finger dragging gesture on one of the active corners of the window.}
\item[B]{The user performs a one finger dragging gesture on a specific tab.}
\item[C]{The user taps a tab to render the window inactive, then performs a two finger pinching gesture anywhere on the window.}
\end{description}

\subsection{Minimizing}
\hfill\\
You are in the middle of writing your email but you need to make a phone call. You decide to minimize the Displex application to hide it and be able to restore it to its previous state after your phone call.
(SHOW PPF)
\\\\
\emph{Open suggestion}
\\\\
\emph{Guided suggestions}
\begin{description}
\item[A]{The user taps a specific tab.}
\item[B]{The user performs a one finger dragging gesture on one of the active corners of the window, reducing it until it becomes an icon.}
\item[C]{The user double taps one of the active corners of the window.}
\end{description}

\subsection{Hiding}
\hfill\\
Later, you are writing another email of a personal nature, and one of colleagues is approaching. You wish to quickly and temporarily hide what you are doing.
(SHOW PPF)
\\\\
\emph{Open suggestion}
\\\\
\emph{Guided suggestions}
\begin{description}
\item[A]
\item[B]
\item[C]
\end{description}

\subsection{Exiting}
\hfill\\
Finally, you are finished and want to leave. You exit the Displex application.
(SHOW PPF)
\\\\
\emph{Open suggestion}
\\\\
\emph{Guided suggestions}
\begin{description}
\item[A]
\item[B]
\item[C]
\end{description}

\end{document}  
