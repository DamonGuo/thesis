\documentclass[11pt]{amsart}
\usepackage{geometry}                % See geometry.pdf to learn the layout options. There are lots.
\geometry{letterpaper}                   % ... or a4paper or a5paper or ... 
%\geometry{landscape}                % Activate for for rotated page geometry
%\usepackage[parfill]{parskip}    % Activate to begin paragraphs with an empty line rather than an indent
\usepackage{graphicx}
\usepackage{amssymb}
\usepackage{epstopdf}
\DeclareGraphicsRule{.tif}{png}{.png}{`convert #1 `dirname #1`/`basename #1 .tif`.png}

\title{Displex early user study}
\author{lnsi}
%\date{}                                           % Activate to display a given date or no date

\begin{document}
\maketitle

\section{Introduction}
``You are here to participate in a short experiment whose purpose is to assist in the design of an application called Displex.
The experiment will last about 30 minutes and is being recorded, both as audio and video.
I will give you instructions by reading from this script.
After an introduction you will be able to ask questions before we start the experiment.

First, let me introduce Displex.
Displex is an application that connects a smartphone (iPhone) with a tabletop computer (Microsoft Surface).
It allows you to interact with your phone on a larger screen, by transferring the display of your phone to a window on the screen of the interactive surface.
Do you understand the basic concept of the application?

During this experiment, I will ask you to perform a task using the application.
This will lead you to perform a number of actions that use the basic features of Displex.
For each action, there will be 3 steps:

\begin{itemize}
\item{First, I will explain the action, and show you its effect on this screen}
\item{Second, I will ask you to describe to me how you would suggest performing this action with the user interface of Displex.}
\item{Third, I will give you 3 suggestions of how to perform the action, and ask you to order them according to your preference.}
\end{itemize}

This experiment is based on prototypes, which means that we will use the available paper representations in order to describe the user interface of Displex.
There are iPhone screenshots and different types of buttons and controls.
Paper, pen and scissors are available for building your own prototypes if necessary.
You are welcome to draw on the prototypes if you want.
We will also use the iPhone, the MS Surface, and of course verbal communication.
\\\\
(EXPLAIN UI PROTOTYPE)
This iPhone screen paper print is a representation of the main window of the Displex application.
The idea is that you can interact with this window in exactly the same way as you would interact with your phone's screen.
For example, by tapping the Photos icon, you would launch the Photos application, and if you are viewing a picture, by performing a two finger pinching gesture, you could zoom on the picture.
At the same time, we need a way to manipulate the window, and that is what this experiment is going to focus on. 
\\\\
Do you have any questions concerning the general course of the experiment?
\\\\
Let us begin.
Your general task is to write an email to a friend using your iPhone and the Displex application on the Microsoft Surface.
We will talk about 7 basic actions.''

\section{Experiment}

\subsection{Pairing}
\hfill\\
This first action is only an example, meaning that I will go through all the steps myself.
The action is called pairing.
\\\\
\emph{Scenario:}
In order to use Displex, I have to pair my iPhone with the Surface and launch the Displex application.
Here is a visual representation of the effect of this application.
(SHOW PPF)
\\\\
\emph{Suggestion:}\\
I asked my advisor Juan, and his suggestion was to launch Displex on the iPhone, then search for available surface computers within the application,  and connect to the Surface.
\\\\
\emph{Selection:}
\begin{description}
\item[A]{The application launches automatically when the smartphone is placed on the surface, and a dialog window appears on the smartphone, offering the user to establish the connection.}
\item[B]{The application launches automatically when the smartphone is placed on the surface, and 2 dialog windows appear, first on the surface, then on the smartphone, offering the user to establish the connection.}
\item[C]{The application launches automatically when the smartphone is close enough to the surface, and a dialog window appears on the surface, offering the user to establish the connection.}
\end{description}

Juans order of preference was B, A, C.
What about you?
\\\\
(SHOW ALL PPF)
There are 6 actions left, and I will now show you visual representations for each of those actions.



\section{Experiment}

\subsection{Pairing}
\hfill\\
This first action is an example, meaning that I will go through all the steps myself.
The action is called pairing.
In order to use Displex, I have to pair my iPhone with the Surface and launch the Displex application. Here is a visual representation of the effect of this application.
(SHOW PPF)
\\\\
\emph{Open suggestion??}\\
I launch Displex on my iPhone, search the local network for available Surface computers, and connect to the Surface.
\\\\
\emph{Guided suggestions}
\begin{description}
\item[A]{The application launches automatically when the smartphone is placed on the surface, and a dialog window appears on the smartphone, offering the user to establish the connection.}
\item[B]{The application launches automatically when the smartphone is placed on the surface, and 2 dialog windows appear, first on the surface, then on the smartphone, offering the user to establish the connection.}
\item[C]{The application launches automatically when the smartphone is close enough to the surface, and a dialog window appears on the surface, offering the user to establish the connection.}
\end{description}

My order of preference is B, A, C.

%%%%%%%%%%%%%%%%%%
% DRAGGING

\subsection{Dragging}
\hfill\\
Your iPhone screen is now active on the surface, and you need to move it closer to yourself. Therefore, you drag the window across the surface.
(SHOW PPF)
\\\\
\emph{Open suggestion}
\\\\
\emph{Guided suggestions}
\begin{description}
\item[A]{By performing a one finger dragging gesture on the active border of the window. (4)}
\item[B]{By performing a one finger dragging gesture on the active border (excl. corners) of the window. (5)}
\item[C]{By holding a finger on a specific tab, and using another finger to tap a destination target to move the window to. (6)}
\end{description}

%%%%%%%%%%%%%%%%%%
% ROTATING

\subsection{Rotating}
\hfill\\
the application window is not oriented correctly, so you need to rotate it to the correct orientation. 
(SHOW PPF)
\\\\
\emph{Open suggestion}
\\\\
\emph{Guided suggestions}
\begin{description}
\item[A]{By performing a two finger touch rotating gesture on the active border. (5)}
\item[B]{By performing a one finger dragging gesture on a corner of the window. (6)}
\item[C]{By performing a two finger touch rotating gesture with one finger placed on a specific tab, and the other anywhere on the window. (1)}
\end{description}

%%%%%%%%%%%%%%%%%%
% RESIZING

\subsection{Resizing}
\hfill\\
Now you open the Safari App by taping on the correct icon, but the window is too small for you to type an email, so you resize it to make it bigger.
(SHOW PPF)
\\\\
\emph{Open suggestion}
\\\\
\emph{Guided suggestions}
\begin{description}
\item[A]{By pulling the window apart with both whole hands. (6)}
\item[B]{By performing a one finger dragging gesture on a specific tab. (1)}
\item[C]{By performing a two finger pinching gesture on the action bar. (2)}
\end{description}


%%%%%%%%%%%%%%%%%%
% MINIMIZING

\subsection{Minimizing}
\hfill\\
You are in the middle of writing your email but you need to make a phone call. You decide to minimize the Displex application to hide it and be able to restore it to its previous state after your phone call.
(SHOW PPF)
\\\\
\emph{Open suggestion}
\\\\
\emph{Guided suggestions}
\begin{description}
\item[A]{By tapping a specific tab. (1)}
\item[B]{By double tapping the action bar. (2)}
\item[C]{By tapping a tab to render the window inactive, then performing a specific gesture anywhere on the window. (3)}
\end{description}

%%%%%%%%%%%%%%%%%%
% HIDING

\subsection{Hiding}
\hfill\\
Later, you are writing another email of a personal nature, and one of colleagues is approaching. You wish to quickly and temporarily hide what you are doing.
(SHOW PPF)
\\\\
\emph{Open suggestion}
\\\\
\emph{Guided suggestions}
\begin{description}
\item[A]{By performing a specific gesture on the action bar. (2)}
\item[B]{By tapping a tab to render the window inactive. (3)}
\item[C]{By double tapping the active border. (4)}
\end{description}

%%%%%%%%%%%%%%%%%%
% EXITING

\subsection{Exiting}
\hfill\\
Finally, you are finished and want to leave. You exit the Displex application.
(SHOW PPF)
\\\\
\emph{Open suggestion}
\\\\
\emph{Guided suggestions}
\begin{description}
\item[A]{By tapping a tab to render the window inactive, then performing a specific gesture on the window. (3)}
\item[B]{By using Minimizing on the active border, then tapping a specific tab. (4)}
\item[C]{By using Minimizing on an active corner, then tapping a specific tab. (5)}
\end{description}

\end{document}  
