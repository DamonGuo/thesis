
%%%%%%%%%%%%%%%%%%%%
%%% INTRODUCTION %%%
%%%%%%%%%%%%%%%%%%%%

%PROBLEM : WHICH INTERACTION TECHNIQUES TO USE WHEN DEVELOPING FOR UI REPLICATION BETWEEN A SMARTPHONE AND A TABLETOP?
%
%SOLUTION : THE PROTOTYPE, SUPPORTED BY BOTH STUDIES

\chapter{Introduction}
\label{introduction}

\section{Research context}

Modern smartphones are performant enough to support most users' daily computing tasks.
They fit in a pocket, which makes them ultra mobile, and they offer good connectivity.
This tendency implies that users have access to personal data and applications at all times.
Smartphones support a new type of computer interaction which is unplanned, spontaneous, on-the-go.
However there are times when the size of the device is a limitation to this form of improvised computer interaction.
This is especially true in situations with multiple users, because a smartphone is too small for several persons to gather around it.

Tabletop computers are cutting-edge devices that merge input and output spaces into one horizontal interactive surface.
They have been the focus of extensive research since the 1990s, with early systems such as the DigitalDesk \citep{Wellner:1993:digitaldesk} and DiamondTouch \citep{Dietz:2001:diamondtouch}, that used overhead projection.
In recent years, tabletops are being commercialized, with interactive displays mostly based on computer vision and capacitive technologies.
Capacitive screens are marketed since the iPhone \citep{iphone}, but are now being produced in larger sizes \citep{displax}, \citep{3m}.
Though its predecessor uses camera-based vision, the latest Microsoft Surface \citep{ms} is based on PixelSense technology \citep{pixelsense}.
Other computer vision based solutions include MultiTaction \citep{multitouch} and side vision overlays \citep{pq}.
Thus, tabletop displays are gradually becoming part of the infrastructure in shared environments such as meeting rooms, public lobbies, bars, restaurants, etc\ldots
They are an ideal platform for spontaneous use, and multi-user interactions.
Furthermore, they support a multi-touch based experience that is in many ways similar to the one on most smartphones.

The latest smartphones boast 720p HD screen resolutions (1280 by 720 pixels) that exceed the naked eye's ability to distinguish separate pixels.
However, sometimes it is the bigger picture that matters.
Reading an article and consulting a map are examples of situations in which small displays present limitations.
To be able to view the data at a convenient scale, the user must zoom in on a portion at a time.
This implies using repetitive zoom and pan gestures, which makes the whole interaction somewhat cumbersome.
In such a situation, a smartphone could benefit from the additional screen space provided by a tabletop.

\emph{Device composition} is a research area that can be traced back to the early work on smart space technologies [REFERENCE].
It has been steadily gaining importance due to the growing multiplicity of computing devices and mobility of users.
Nowadays, a typical user owns mobile computers (handheld, tablet, laptop, etc\ldots) and interacts with other devices in an ad hoc way, as s/he comes upon them throughout the day (public desktops, printers, displays, etc\ldots).
Enabling efficient communication and collaboration between various devices is therefore an essential issue, with the overall goal of improving the user experience.
\\\\
Device composition focuses on the following challenges:

\begin{description}
\item[Connection:] this includes device detection, identification and connection.
\item[Communication:] different types of devices (hardware, OS) must use a common language if they are to collaborate.   
\item[Sharing:] collaboration often requires sharing information. Besides technical problems, this raises the privacy issue of protecting the user's personal data.
\item[Interaction:] the user must be able to interact with the system if s/he is to benefit from it.
\end{description}
\hfill\\
This project focuses on the human computer interaction aspect of combining smartphones and tabletops.
The slightly broader issue of combining smartphones and larger interactive displays has been approached in various ways, each focusing on a specific interaction metaphor.
\begin{description}
\item[Streaming] is a one way approach where only the visual output of the smartphone is forwarded to a remote display.
\item[Replication] goes one step further, by allowing the user to interact with the replicated UI.
\item[Projection] is a metaphor similar to replication, in which the smartphone is used as a projector, allowing the user to ``drop'' the UI onto an available display \citep{Winkler:2011:interactivephonecall}.
\item[Adaptation] refers to an improved UI transfer where the UI is modified to make full use of the additional resources offered by the remote display \citep{Arthur:2011:xice}.
\item[Extension] provides the possibility of transferring single applications/processes to a remote  machine.
\end{description}

This project focuses on UI replication because it improves the user experience while keeping the interaction natural, and it can be implemented with the available resources.
On a technical level, the advantage of UI replication is that it uses the application logic of the personal device, requiring of the remote display only to forward graphical output, and touch-based input.
This allows the development of software that is easily adaptable to various programming platforms.
On a human-computer interaction level, it reduces the learning curve for the user, by providing an intuitive experience that is similar to the one s/he is used to.
By comparison, the streaming metaphor is too limited, and the other paradigms all introduce new interaction dimensions that require user adaptation.
A final argument in favor of UI replication is that it allows the implementation of an engaging prototype without requiring any additional graphical design.

\section{Problem statement}

\emph{Intuition} is the ability to understand something immediately, without the need for conscious reasoning.
\\\\
This project attempts to find out \emph{how to design intuitive systems that integrate smartphones and tabletops.}
\\\\
Particularly, the following questions are asked:
\begin{itemize}
\item is it feasible to build a system that supports the UI replication of any smartphone to the Microsoft Surface tabletop computer?
\item which interaction techniques should be used to design for an intuitive user experience?
\end{itemize}

\section{Research methods}

To answer these questions, the following methods are used:
\begin{itemize}
\item a literary review and analysis of the research background,
\item a requirements analysis of the system,
\item a solution design produced via a user-centered approach \citep{Benyon:2010},
\item the implementation of an application prototype,
\item an evaluation of the solution by way of a usability study.
\end{itemize}

\section{Results}

This report shows that it is possible to develop software that supports novel interactions, without requiring any conscious learning effort of the user, by designing intuitive systems.
\emph{TIDE} (Tabletop Interactive Display Extension) is a prototype that makes using a tabletop to interact with a smartphone as natural as interacting with the smartphone itself.
The report presents design guidelines with a list of the interaction techniques that are most intuitive for the user in this application context.

The TIDE prototype shows that it is possible to implement an application on the Microsoft Surface that replicates the UI of any type of smartphone.
It currently supports iOS and Android smartphones, and can easily be extended to other devices.
The system consists of two main components.
The \emph{Remote UI} is a window on the tabletop that replicates the UI of a smartphone, and allows remote interaction.
It is based on the VNC protocol \citep{Richardson:1998:vnc}.
The \emph{Surface UI} is the tabletop UI that contains the Remote UI, and provides controls to manipulate it.

\section{Thesis overview}

Chapter~\ref{relatedwork} presents a literary review of the research that constitutes the background to this work, and the theoretical work on which the design approach is based.\\
Chapter~\ref{design} describes the process that lead to the design of the TIDE prototype.\\
The system itself is presented in Chapter~\ref{system}, and its evaluation by way of a usability study in Chapter~\ref{evaluation}.\\
Chapter~\ref{discussion} is a discussion that addresses the results and lessons learned throughout this process, and brings suggestions for future work.\\
Chapter~\ref{conclusion} concludes the report.

% tabletop application/research examples
%%%%%%%%%%%%%%%%%
%Tabletop computers are cutting-edge devices that merge input and output spaces into one single interactive surface \cite{Wellner:1993:digitaldesk}.
%Researchers have investigated the use of interactive tables in a number of different ways: support for meetings, canvas for architectural design \cite{Clifton:2010:sketchtop}, media for document navigation, mediator for sharing files, etc.
%Due to their size and embedded nature, tabletops seem to naturally fit in public spaces such as shops, bars and work places.
%Common scenarios include catalog browsing, drink ordering and product configuration.
%Technologies such as DiamondTouch \cite{Dietz:2001:diamondtouch} allow tabletops to support multiple and simultaneous users. Example of applications include sharing data between smartphones, collaborating on a design \cite{Hunter:2011:memtable}, or simply taking notes during a meeting.
%In the case of multiple individualized users, solutions are needed to identify each user, as seen in \cite{Schmidt:2010:handsdown}, where the simple action of placing one's hand on the surface enables a person to identify and start interacting with the device.

% tabletop integration with tangibles
%%%%%%%%%%%%%%%%%%%%%%%%%%%%%%%%%%%%%%
%Another interesting property of interactive surfaces is their ability to integrate with physical objects, both passive and dynamic, for the purpose of augmenting them with digital information, or controlling the application state.
%For example, SurfaceWare \cite{Dietz:2009:surfaceware} allows the Microsoft Surface to sense the fluid level in a slightly enhanced drinking glass.
%Another example is the software developed by Amnesia Razorfish, that allows the sharing of data between multiple handheld devices using the actual devices, as well as gestures, on the Microsoft Surface.
%Finally, researchers at ITU have developed the Rabbit \cite{Hincapie:2011:rabbit}, a device that integrates small RFID-tagged objects and tabletops.

% solution : using tabletops as UI peripherals
%%%%%%%%%%%%%%%%%%%%%%%%%%%%%%%%%%%%%%
%The specificity of tabletops raises the question of how to interact with them on an everyday basis.
%Recent development initiatives tend to answer this question by regarding tabletops as yet another computational platform, requiring its own software.
%With this project, we explore a different approach to integrating tabletops in our environment, namely by using them only as UI peripheral, providing touch-based input and graphical output to the devices that we already have.
%Exploring this path is supported by three important factors.
%First, most users already own computing devices, such as laptops or smart phones, with tailor-made applications and local storage, and might be less prone to use an additional device if it requires management (updates, backups, synchronizations, etc) and the purchase of applications.
%Second, tabletops are embedded in the environment and as such can be expected to be shared devices.
%Using them as simple graphic peripheral would allow to avoid the traditional desktop/laptop issues related to user profiles, privacy and data integrity.
%Finally, as embedded devices, it is reasonable to expect tabletops to have good networking capabilities.

% device composition
%%%%%%%%%%%%%%%%%%%%%%%%%%%%%%%%%%%%%%
%Device composition focuses on getting the most out of various computing entities, by making them work together and function as one, as seen in \cite{Bardram:2010:compute}.
%This project explores device composition for UI integration between tabletops and mobile devices, focusing on seamless user experience and implicit human computer interaction as defined by Schmidt in \cite{Schmidt:2000:implicit}.

% UI integration metaphors
%%%%%%%%%%%%%%%%%%%%%%%%%%%%%%%%%%%%%%
%UI integration can happen in several different ways:
%\begin{itemize}
%\item{\emph{UI transfer} (mirror): the tabletop `takes over' and displays the UI of the connected device.}
%\item{\emph{Dual view}: the tabletop display becomes secondary screen space for the connected device.}
%\item{\emph{UI nesting}: the connected device is physically located on the tabletop, and its UI is extended to the additional screen space around it.}
%\end{itemize}

% challenges
%%%%%%%%%%%%%%%%%%%%%%%%%%%%%%%%%%%%%%
%Following is an open list of problems that we will address in order to achieve device composition by means of implicit interaction.
%\begin{enumerate}
%\item{\emph{Setup}: How is a device enabled for integrating with a tabletop?
%The setup should be simple, to be performed only once by non-technical users.
%An initial survey of possible solutions points towards the use of tagging mechanisms and/or camera-based object recognition.}
%\item{\emph{Discovery}: How do the tabletop and the device discover and communicate with each other?
%How do we solve the issues of discovery, handshake, network connectivity, and encryption mechanisms to ensure privacy?}
%\item{\emph{UI transfer}: Given the computational constraints of mobile devices, how can the UI transfer be efficiently implemented so as to support native applications and guarantee a seamless user experience?}
%\item{\emph{Input}: How can the users interact with their applications on the tabletop (touch and other peripherals)?}
%\item{\emph{Interaction Design}: What means of interaction are best-fitted for the tabletop-based systems that we propose to develop?
%How can we best adapt to public/private uses and single/multiple users?
%How can we take advantage of the larger interaction surface?}
%\end{enumerate}

