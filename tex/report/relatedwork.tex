\chapter{Related Work}
\label{relatedwork}

This chapter describes the research context in which this work was conducted, by reviewing prior work belonging to five categories.

\emph{Device composition}, presented in section~\ref{sec:rwcomposition}, is the main background of this thesis.
Combining smartphones and tabletops is the approach that was followed to solve the problem at hand, and is reviewed in section~\ref{sec:rwintegration}.

Section \ref{sec:rwpairing}, \ref{sec:rwdistribution} and \ref{sec:rwinteraction} review the existing solutions to three fundamental challenges of device composition, i.e.\ respectively;
(1) \emph{Pairing}, the procedure through which devices associate with each other
(2) \emph{UI distribution}, the technological approach to distributing user interfaces between devices
(3) \emph{User interaction}, the type of metaphor that is used when combining devices for UI interoperation.

\section{Device composition}
\label{sec:rwcomposition}

%DEVICE COMPOSITION is an approach that can solve this.
%%%%%%%%%%%%%%%%%%%%%%%%%%%%%%%%%%%%%%%%%%%%%%%%%%%%%%%%%%%%%%%%%%%
Device composition is an important element of the ubiquitous computing research area, and the basis of this thesis.
An essential aspect of ubiquitous computing is the idea that users can benefit freely from the resources that are available in the environment.
%Device composition started with SMART ROOMS, and composing various devices, which we want to do in this case.
%But smart rooms were designed as closed environments.
%%%%%%%%%%%%%%%%%%%%%%%%%%%%%%%%%%%%%%%%%%%%%%%%%%%%%%%%%%%%%%%%%%%
Extensive research focused on realizing that idea by designing computer augmented spaces, or \emph{smart rooms}, where heterogeneous devices can interoperate.
Examples of such smart spaces are Augmented Surfaces \citep{Rekimoto:1999:augmentedsurfaces}, 
%where portable devices and interactive surfaces are enabled for data exchange; 
\mbox{i-LAND} \citep{Streitz:1999:iland} and the Interactive Workspaces \citep{Johanson:2002:iroom}.
Beside device interaction, these projects were aimed at supporting collaborative work between collocated users.
Smart rooms are usually closed computing environments that rely on a centralized software infrastructure, such as BEACH developed for i-LAND \citep{Tandler:2001:smartenv}, or Gaia OS for Active Spaces \citep{Roman:2002:gaia}.
The advantage of working with a dedicated software platform is that the interaction between the enabled devices can be optimized, as they share a set of semantics.
However, the drawback is that new devices cannot be integrated easily, as they require a specific software configuration.

%We would like AD-HOC, spontaneous device composition.
%That was addressed by projects like Obje, CompUTE, Platform Composition.
%%%%%%%%%%%%%%%%%%%%%%%%%%%%%%%%%%%%%%%%%%%%%%%%%%%%%%%%%%%%%%%%%%%
To bring device composition out of the smart rooms, systems that support a more ad-hoc form of device interaction have been built.
Obje \citep{Edwards:2009:obje} is a middleware that relies on services transferring mobile code to achieve device compatibility at runtime.
It allows devices to interoperate upon their very first encounter, but relies on the user's semantic interpretation to do so.
\cite{Pering:2009:platformcomp} focused on understanding ad-hoc collaboration using pervasive technologies, and suggested Platform Composition as a technique to support collaborative work by using standard computing components.
\cite{Bardram:2010:compute} developed CompUTE, a runtime architecture for device composition on Windows XP systems based on the extended desktop metaphor.

Tabletop displays are a recurring element in research projects that address device composition.
They have two important characteristics.
First, they are situated.
Due to their size, they typically sit in a location and are not moved frequently.
Second, they are shared among multiple users.
Their horizontal interactive surfaces allow users to attend from all angles and to engage in face-to-face interaction.
For these reasons, tabletops seem to perform the best in combination with the personal devices carried around by users.

In recent years, tabletops are being commercialized, and gradually appear in public spaces such as firm lobbies, shops, bars and restaurants.
For example, the Microsoft Surface is part of the furniture in the new retail stores of the Royal Bank of Canada \citep{mscase}, offering service applications to the visiting customers.
This tendency gives rise to a new type of tabletop interaction, more spontaneous, and targeting non technical users.
It is within this context that the present work is set.

\section{Integrating smartphones with tabletops}
\label{sec:rwintegration}

Due to the horizontal orientation of tabletop displays, they are an ideal platform for integration with the things that we naturally place on them.
Much work has gone into the detection and tracking of material objects, with systems such as SurfaceFusion \citep{Olwal:2008:surfacefusion}, where the authors used RFID technology to sense the presence of objects, and computer vision to track their precise location.

Thus, computer vision can also be used to track other computing devices, such as smartphones.
This approach has been followed by projects such as the BlueTable \citep{Wilson:2007:bluetable}, and the work by \cite{Echtler:2008:tracking}, where mobile phones are connected to a tabletop via Bluetooth, and the interaction is enhanced by having the tabletop track the location of the devices.

By using a personal device such as a mobile phone to interact with a tabletop, it is possible to provide user-identification to the interaction.
An example of this is PhoneTouch \citep{Schmidt:2010:phonetouch}, where the tabletop and a connected phone separately detect a touch event, and by comparing timestamps identify the phone.
This technique can be used for personalization of the interface and user-authentication.
It was also used by \cite{Berglund:2011:nai}, who implemented a framework to develop applications where smartphones can provide user-identification when interacting with a tabletop.

Another scenario for which device composition between smartphones and tabletops is promising, is the viewing and sharing of images.
This has been shown by projects such as Throw Your Photos \citep{Chehimi:2010:throwphotos} and Pour Images \citep{Esbensen:2010:pourimages}, where digital images can be transferred to a tabletop via a Bluetooth connection, and be viewed, edited and shared with other users.

%tables are good for collaboration
%\citep{Shen:2007:clicks}
%
%browsing, sorting and sharing digital images on interactive surfaces (with physical control object)
%\citep{Hilliges:2007:photohelix}


%Even though early systems such as the DigitalDesk \citep{Wellner:1993:digitaldesk} were designed for the individual office desk, most of the research on tabletops has been focused on their potential as support for collocated collaboration.
%Thanks to technology advances such as DiamondTouch \citep{Dietz:2001:diamondtouch},
%% that combines overhead projected output with capacitively sensed input,
%tabletops support multiple simultaneous touch input, and thus simultaneous users.
%They are an essential element of smart spaces, with systems such as the InteracTable for the i-LAND \citep{Streitz:1999:iland} and the iTable for the iRoom \citep{Johanson:2002:iroom}.
%Other projects have shown that they are an ideal tool to support collaboration, with systems such as the MemTable \citep{Hunter:2011:memtable}, that permits the capture and recall of meetings, or SketchTop \citep{Clifton:2010:sketchtop}, a multitouch sketching application for collocated design collaboration.

% tangible interaction %
%The integration of innate objects to tabletop systems has given rise to a research branch called \emph{tangible interaction} ..
%
%There is basically two ways to integrate physical objects to interactive surface.
%The first one is to use the objects as a tangible user interface (TUI), a notion that was introduced with the MetaDESK \citep{Ullmer:1997:metadesk}.
%The second approach has for purpose to augment objects with digital information.
%Examples of this approach include SurfaceWare \citep{Dietz:2009:surfaceware}, which allows the Microsoft Surface to sense the fluid level in a slightly enhanced drinking glass, and the Rabbit \citep{Hincapie:2011:rabbit}, a device that integrates small RFID-tagged objects and tabletops.

\section{Pairing}
\label{sec:rwpairing}

Pairing is the procedure through which two devices first discover each other, then connect to each other for interaction.
It is an essential part of any device composition, and a challenge that can be solved in many ways.

Some systems are designed for networked devices, and achieve pairing with TCP/IP based networking protocols such as Zeroconf \citep{zeroconf} for Obje and UPnP \citep{upnp} for CompUTE.

Bluetooth \citep{bluetooth} supports a device discovery protocol that was used by the Personal Server \citep{Want:2002:personalserver}.
It allows the pairing of any bluetooth-enabled device, but the process has been shown to be slow, and can be inefficient if multiple devices are in the vicinity.

In systems that allow for spontaneous interaction between wireless mobile devices, additional techniques must be used to permit the identification of the correct device.
A way to do that is by detecting synchronous events, as with the Smart-its \citep{Holmquist:2001:smartits}, that connect when they are held and shaken together; and SyncTap \citep{Rekimoto:2003:synctap}, that require the same key to be simultaneously pressed on both devices.

Another technique that has been used to connect mobile devices to a situated one, e.g.\ a large display, is having the display present a random key that has to be entered on the mobile device.
The key can have the form of an alphanumeric string, a sequence of motions \citep{Patel:2004:mobileauth}, or a visual pattern \citep{Ballagas:2005:sweeppointshoot, Scott:2005:visualauth}.
This approach allows authentication, but adds steps to the pairing process.

Established RFID technologies can be used for device association, but requires the mobile device to be equipped with the appropriate tag, as well as the situated device presenting an RFID reader.
Upcoming NFC techniques present the great advantage that devices can function both as reader and transmitter, but few commercial devices are equipped as of yet.

Computer vision can be used to detect specific shapes, such as phone-like objects.
It can make the pairing process easier, as shown with the BlueTable \citep{Wilson:2007:bluetable}.
%\\
%\linebreak

\section{UI distribution}
\label{sec:rwdistribution}

Device composition has been extensively used for building systems that allow the sharing of user interfaces between devices.
The typical setup is to be able to view and interact with the data and applications from a mobile device on a situated one that offers superior display resources.
There are various technologies for UI distribution.

One way is to distribute only the data to the larger display, such as with the iRoom \citep{Johanson:2002:iroom}, however this approach is only applicable when both devices have the same software installed.

Another possibility is to distribute code, sending whole applications to the situated device for execution.
However, there is a series of issues with such an approach.
The situated device might not be able to run the application, either because of compatibility problems (hardware, software) or because the device might choose not to trust unknown code.
Software can be compiled into platform-independent code, such as with Flash \citep{flash}, Java \citep{java} and Silverlight \citep{silverlight}.
However, issues that are mostly related to the excessive size of the data files and the security risks still remain.
\\
\linebreak
Distributing graphics is generally a preferred approach.
It allows for user data and credentials to be kept on the mobile device, and it has better potential for responsive interaction.
Systems such as X-Windows (X11) \citep{Scheifler:1986:x11}, Remote Desktop Protocol (RDP) \citep{Tritsch:2003:rdp} and Virtual Network Computing (VNC) \citep{Richardson:1998:vnc} utilize rendering-based protocols for UI distribution.
Applications are executed on one device, and rendered on another.
These protocols are stable, but they are based on the assumption that the machine executing the application has important resources.
When combining mobile and situated devices, the situation is however reversed; the computer with the greater resources renders the graphics.

Another way of distributing graphics is by using web-based protocols, where Hypertext Transfer Protocol (HTTP) and Hypertext Mark-up Language (HTML) are used to send and present the application.
Such an approach can either be built on a web server model, which is basically what we use when we log on to websites via a browser, or on a personal server model, which is what the Personal Server \citep{Want:2002:personalserver} was based on.
The Personal Server is a handheld device without a display, that transmits HTML pages to other available displays in the environment.
Using a web-based protocol implies a series of drawbacks that are induced by the use of HTML; browsers behave differently, HTML 1 to 4 does not support generalized drawing, HTML does not handle varying display sizes and resolutions, and the use of Javascript introduces security issues related to programmable display.

Recent research efforts have been invested in the development of whole frameworks that can support flexible UI distribution.
This is the case with XICE (eXtending Interactive Computing Everywhere) \citep{Arthur:2011:xice}, a programming framework that uses wireless networks to connect portable devices to display servers.
It takes the limited CPU and battery capacities of mobile devices into account, and provides a flexible protocol that allows for annexation of different types of displays.

Substance \citep{Gjerlufsen:2011:substance} was designed to support the development of interactive multi-surface applications.
It is a data-oriented programming model that was used to build Shared Substance; a middleware that provides powerful sharing applications.

%X-based window system to facilitate the design, implementation and evaluation of innovative window management techniques \citep{Chapuis:2005:metisse}.

% TRACKING USERS
%survey of advances in vision-based human motion capture and analysis between 2000 and 2006, show substantial progress with automatic human motion tracking and recognition.
%\citep{Moeslund:2006:motioncapture}

%Device composition has been steadily gaining importance due to the growing multiplicity of computing devices and mobility of users.
%Nowadays, a typical user owns mobile computers (handheld, tablet, laptop, etc\ldots) and interacts with other devices in an ad hoc way, as s/he comes upon them throughout the day (public desktops, printers, displays, etc\ldots).
%Enabling efficient communication and collaboration between various devices is therefore an essential issue, with the overall goal of improving the user experience.

%Device composition focuses on the following challenges:
%\begin{description}
%\item[Connection:] this includes device detection, identification and connection.
%\item[Communication:] different types of devices (hardware, OS) must use a common language if they are to collaborate.   
%\item[Sharing:] collaboration often requires sharing information. Besides technical problems, this raises the privacy issue of protecting the user's personal data.
%\item[Interaction:] the user must be able to interact with the system if s/he is to benefit from it.
%\end{description}
%\hfill\\
%This project focuses on the human computer interaction aspect of combining smartphones and tabletops.

\section{User interaction}
\label{sec:rwinteraction}

%- intro / theory = Norman, Buxton (sketching user experiences)
%- methods / approaches
%- gestures
%- interactions

Combining mobile devices with larger displays is not a new idea.
It is already widely available for laptop computers, that can be connected to external monitors or projectors via a cable such as VGA or DVI.
There are serious limitations to this type of setup.
Some are technical; said cables are limited in terms of graphical output, the large size of the connectors is prohibitive on mobile devices.
Others are related to the user experience; the user must have the required cable, the user must find the connector on the larger display, the process of connecting/disconnecting is susceptible to take time.
This thesis focuses therefore on providing a wireless user experience.

Recent research efforts show different approaches, to improving the experience of the smartphone user, by allowing the extension of the phone UI to an available display surface.
The main approaches are listed here in terms of the interaction metaphor that they are based on.

\emph{Streaming} is a straightforward approach where only the visual output of the mobile device is forwarded to a remote display.
It is what happens when we connect a laptop to a projector, e.g.\ to give a presentation.
This is a satisfactory solution in specific situations, and presents the advantage of being secure, since the input is kept to the mobile device.
However, it presents serious limitations in terms of the interaction potential.

\emph{Replication} goes one step further, by allowing the user to provide input on the replicated UI.
This type of integration is already available for personal computers, but requires a cable, or a docking station.

\emph{Expansion} is when the larger display provides additional space for the existing UI of the mobile device.
It is also available for laptops via a cable or docking station.
It provides the possibility of transferring application windows to the remote display, while preserving the integrality of the mobile UI.

\emph{Projection} is a metaphor similar to replication, with the difference that the mobile device itself is used for projecting its UI onto an available display surface.
\cite{Winkler:2011:interactivephonecall} have built a device composed of a smartphone and a personal projector, that provide powerful in-call collaboration features, by projecting a shared interface on any available surface.
Virtual Projection (VP) \citep{Baur:2012:virtualprojection} is a system where a smartphone can be used to project its UI onto an available display.

\emph{Adaptation} refers to an improved UI transfer where the UI is modified to make full use of the available resources on the remote display.
This approach was followed by the authors of XICE \citep{Arthur:2011:xice}.
\\
\linebreak
In comparison to prior work, this thesis focuses on learnability and ease of use.
The goal is to design a system that can support spontaneous interaction with non technical users successfully, by implementing interaction techniques that are consistent with the smartphone user experience.
Additionally, this work examines the use of standard hardware and protocols to build a system that can be adapted to multiple platforms, and is compatible with various smartphone types.

